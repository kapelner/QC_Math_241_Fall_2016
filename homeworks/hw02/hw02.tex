\documentclass[12pt]{article}

\include{preamble}

\newtoggle{spacingmode}
\toggletrue{spacingmode}  %STUDENTS: DELETE or COMMENT this line

\newtoggle{professormode}
\toggletrue{professormode} %STUDENTS: DELETE or COMMENT this line

\newcommand{\spc}[1]{\iftoggle{spacingmode}{\\ \vspace{#1cm}}}



\title{MATH 241 Fall 2016 Homework \#2 \\ \inred{EXPERIMENTAL --- DO NOT PRINT}}

\author{Professor Adam Kapelner} %STUDENTS: write your name here

\iftoggle{professormode}{
\date{Due 5PM outside my office KY604, Tuesday, September 13, 2016 \\ \vspace{0.5cm} \small (this document last updated \today ~at \currenttime)}
}

\renewcommand{\abstractname}{Instructions and Philosophy}




\begin{document}
\maketitle

\iftoggle{professormode}{
\begin{abstract}
The path to success in this class is to do many problems. Unlike other courses, exclusively doing reading(s) will not help. Coming to lecture is akin to watching workout videos; thinking about and solving problems on your own is the actual ``working out''.  Feel free to \qu{work out} with others; \textbf{I want you to work on this in groups.}

Reading is still \textit{required}. For this homework set, read the section about sample spaces in Chapter 2 and relevant parts of Chapter 1 in Ross. Chapter references are from the 7th edition.

The problems below are color coded: \ingreen{green} problems are considered \textit{easy} and marked \qu{[easy]}; \inorange{yellow} problems are considered \textit{intermediate} and marked \qu{[harder]}, \inred{red} problems are considered \textit{difficult} and marked \qu{[difficult]} and \inpurple{purple} problems are extra credit. The \textit{easy} problems are intended to be ``giveaways'' if you went to class. Do as much as you can of the others; I expect you to at least attempt the \textit{difficult} problems.

This homework is worth 100 points but the point distribution will not be determined until after the due date. See syllabus for the policy on late homework.

Up to 15 points are given as a bonus if the homework is typed using \LaTeX. Links to instaling \LaTeX~and program for compiling \LaTeX~is found on the syllabus. You are encouraged to use \url{overleaf.com}. If you are handing in homework this way, read the comments in the code; there are two lines to comment out and you should replace my name with yours and write your section. If you are asked to make drawings, you can take a picture of your handwritten drawing and insert them as figures or leave space using the \qu{$\backslash$vspace} command and draw them in after printing or attach them stapled.

The document is available with spaces for you to write your answers. If not using \LaTeX, print this document and write in your answers. I do not accept homeworks not on this printout. Keep this first page printed for your records. Write your name and section below (A or B).

\end{abstract}

\thispagestyle{empty}
\vspace{1cm}
NAME: \line(1,0){240} ~~SECTION (A or B): \line(1,0){35}
\pagebreak
}


\iftoggle{professormode}{
\paragraph{More counting} These counting questions will give you more practice in computing probabilities. Due to computations involving large factorials, we will also review Stirling's Approximation.\\ \\
}


\problem We have 4 blue marbles, 4 green marbles, 2 orange marbles, and 2 red marbles. For the following questions, if you are using ``choose notation'', please write your choose notation, then write the formula using factorials, then write the actual number after you compute it.

\iftoggle{professormode}{
\begin{figure}[htp]
\centering
\includegraphics[width=2.1in]{marbles.jpg}
\end{figure}
\FloatBarrier
}

\begin{enumerate}
\easysubproblem Viewing all the marbles as \textit{unique}, how many ways are there to order the marbles? Note that \qu{order} is another way of saying \qu{permute.}\spc{2}

\intermediatesubproblem Viewing all marbles of the same color as \textit{interchangeable}, how many ways is there to order the marbles? \spc{2}

\extracreditsubproblem If I pick 4 marbles at random from the collection, how many ways are there to get two-of-a-kind \ie two marbles of one color and two marbles of a different color. \spc{7}

\end{enumerate}

\problem Imagine you have a bag of 10 cards where 6 are blue and 4 are red. A \qu{draw} means one card is taken out of the bag at random and the color is revealed. If the problem asks \qu{what is the probability,} this means an explicit computation is required unless otherwise stated.

\iftoggle{professormode}{
\begin{figure}[htp]
\centering
\includegraphics[width=1.2in]{red_blue.png}
\end{figure}
\FloatBarrier
}

\begin{enumerate}
\easysubproblem What is the probability of getting a blue card when drawing one card? \spc{1}

\easysubproblem What is the probability of drawing 3 red cards in a row \textit{without replacement}? \spc{1.5}

\intermediatesubproblem Five cards are drawn. What is the probability of having 3 reds and 2 blues without regards to any order of the cards? \spc{2}

\hardsubproblem Five cards are drawn. What is the probability of having 3 reds and 2 blues in that order? Think carefully about the numerator and denominator in this probability computation. \spc{4}

\intermediatesubproblem Five cards are drawn from a new bag with 100 cards where 60 are blue and 40 are red. What is the probability of having 3 reds and 2 blues without regards to any order of the cards? \spc{3}

\intermediatesubproblem Five cards are drawn from a new bag with 1000 cards where 600 are blue and 400 are red. What is the probability of having 3 reds and 2 blues without regards to any order of the cards? \spc{4}

\end{enumerate}

\problem Imagine you are putting together musical performances and you are employing musicians at random. There are many available for hire: 23 guitarists, 15 vocalists, 6 drummers, 14 bassists, 8 violinists, 9 violas, 6 cellists.


\iftoggle{professormode}{
\begin{figure}[htp]
\centering
\includegraphics[width=2in]{musicians.png}
\end{figure}
\FloatBarrier
}

\begin{enumerate}
\easysubproblem If we hire 4 musicians at random, what is the probability we get a rock band (a vocalist, a guitarist, a bassist and a drummer)? \spc{3}

\easysubproblem If we hire 4 musicians at random, what is the probability we get a string quartet (two violinists, 1 cellist and one violist)? \spc{3}

\easysubproblem If we hire 4 musicians at random, what is the probability we get a doowop group (four vocalists)? \spc{7}

\easysubproblem We now move to a different city and the musicians for hire are different. Here, we have 10 guitarists, 10 vocalists, 10 drummers, 10 bassists. What is the probability we form a rock band when hiring four musicians at random? \spc{3}

\hardsubproblem Given the same situation in part (d), what is the probability we get two pairs of musicians (e.g. two guitarists and two bassists or two drummers and two bassists)? \spc{2.8}

\hardsubproblem Given the same situation in part (d), what is the probability we get all four musicians be the same type? \spc{3}

\end{enumerate}

\problem Computations of combinations and permutations is impossible for a computer due to the factorial computations. In this problem, we investigate Stirling's formula

\begin{enumerate}
\easysubproblem Use the natural log to derive an expression for $\binom{1500}{300}$ using sums by recalling that $\natlog{n!} = \sum_{i=1}^n \natlog{i}$. Do not compute. Leave in terms of sums and natural logs. \spc{3}

\easysubproblem Show that Stirling's approximation is equivalent to the expression I wrote in class:

\beqn
\natlog{n!} \approx \half\natlog{2\pi} + \parens{n + \half}\natlog{n} - n
\eeqn
\spc{4}

\hardsubproblem Use the expression in (c) to approximate the probability of getting 300 Heads in 1500 coin flips. \spc{6}

\end{enumerate}

\problem Combinations are not only useful in probability problems. They come up all over mathematics.

\begin{enumerate}
\intermediatesubproblem In the first lecture we mentioned that $\abss{2^\Omega} = 2^{|\Omega|}$. (recall that the powerset contains all subsets of $\Omega$ \ie $A \in 2^\Omega~~\forall A \subseteq \Omega$). We reasoned that each $\omega \in \Omega$ can be either \textit{in} or \textit{out} of a subset. Thus on/off for the first outcome, on/off for the second outcome, etc. to make 2 raised to the number of elements. This will count every possibly subset. All \qu{offs} would result in $\varnothing$ and all \qu{ons} will result in $\Omega$. 

Prove the following using notes from class:

\beqn
\abss{2^\Omega} = \sum_{i=0}^{|\Omega|} \binom{|\Omega|}{i}
\eeqn
\spc{6}

\easysubproblem  Recall the binomial theorem:

\beqn
(a + b)^n = \sum_{i=0}^n \binom{n}{i} a^{n-i} b^i
\eeqn

Explain \textit{in English} why are the $\binom{n}{i}$ terms called \qu{binomial coefficients.} \spc{4}

\extracreditsubproblem Prove that $\binom{n}{i} = \binom{n-1}{i} + \binom{n-1}{i-1}$. You need to understand the class lecture here. \spc{8}


\easysubproblem Below is \qu{Pascal's Triangle} up to $n=4$.

\begin{table}[htp]
\centering
\begin{tabular}{rccccccccc}
$n=0$:&    &    &    &    &  1\\\noalign{\smallskip\smallskip}
$n=1$:&    &    &    &  1 &    &  1\\\noalign{\smallskip\smallskip}
$n=2$:&    &    &  1 &    &  2 &    &  1\\\noalign{\smallskip\smallskip}
$n=3$:&    &  1 &    &  3 &    &  3 &    &  1\\\noalign{\smallskip\smallskip}
$n=4$:&  1 &    &  4 &    &  6 &    &  4 &    &  1\\\noalign{\smallskip\smallskip}
\end{tabular}
\end{table}

Explain why the 6 in the middle of the $n=4$ row is equivalent to $\binom{4}{2}$ by using the fact from part (c). \spc{2}

%\extracreditsubproblem Prove the equality in part (a) for arbitrary but finite-sized $\Omega$. \spc{9}

\extracreditsubproblem Prove the binomial expansion in part (b) for arbitrary $n \in \naturals$. \spc{9}

\end{enumerate}

%\problem This problem involves using the multinomial coefficient to solve problems.
%
%\begin{enumerate}
%\easysubproblem Imagine you have 12 flowers: 4 red and 3 blue and 5 white. How many ways are there to arrange them in 12 flower pots. \spc{1.5}
%
%\easysubproblem We add 2 orange flowers to collection in part (a). How many ways to arrange the flowers now? \spc{1.5}
%
%\easysubproblem Imagine we have 5 flowers: one white, one blue, one red, one orange and one purple. How many ways to arrange them? Use the multinomial coefficient and show that it is equal the number you arrive at using the permutation concept from lecture 2. \spc{2.5}
%
%\end{enumerate}

%\iftoggle{professormode}{
%\paragraph{Probability as Applied Set Theory} Problems below are related to set theory and probability\\ \\
%}
%
%\problem We will get our feet wet with basic \qu{axioms} and theorems. Assume all capital letters are sets. If the problem asks you to prove a fact, you may only use your knowledge of set theory and the definition of $\prob{\cdot}$ given in the book / lecture. Some of the answers are in the book. Try to do them yourself and only use the book if you are having trouble. The extra credits are really, really difficult.
%
%\begin{enumerate}
%\easysubproblem List all assumptions prior to and the three conditions that make $\prob{\cdot}$, the set function that returns a probability. These three conditions are also known as the \qu{axioms of probability.} \spc{3.5}
%
%\easysubproblem Prove that if $A_1$ and $A_2$ are disjoint (mutually exclusive), $\prob{A_1 \cup A_2} = \prob{A_1} + \prob{A_2}$. \spc{1.5}
%
%\easysubproblem Prove that $\prob{\varnothing} = 0$. \spc{2.5}
%
%\intermediatesubproblem Prove that $\prob{A} \in \zeroonecl$. \spc{5.5}
%
%\hardsubproblem Prove that if $A \subseteq B$ then $\prob{A} \leq \prob{B}$. \spc{3.5}
%
%\hardsubproblem Prove that $\prob{A \cup B} = \prob{A} + \prob{B} - \prob{A, B}$. \spc{4.5}
%
%\extracreditsubproblem Describe a sequence of sets $A_1, A_2, \ldots$ which are all non-empty where $\sum_{i=1}^\infty \prob{A_i} = 1$. \qu{Describe} means to explicitly state the elements in each of the sets. Hint: the sets do not have to be finite nor countable for that matter. I strongly suggest you also construct $A_1, A_2, \ldots$ as disjoint otherwise the sum of their probabilities may be greater than 1. \spc{8.5}
%
%\extracreditsubproblem Let $A_1 \subseteq A_2 \subseteq A_3 , \subseteq \ldots$ (this is called a sequence of \qu{increasing events.}) Prove that:
%
%\beqn
%\limitn \prob{A_n} = \prob{\limitn A_n}
%\eeqn
%
%~\spc{7.5}
%
%\end{enumerate}

\iftoggle{professormode}{
\paragraph{Philosophy of Probability} We began philosophy of probability in last class and will continue next lecture.\\ \\ 
}

\problem Answer the following questions by writing a paragraph or two \textit{in English}.

\begin{enumerate}
\easysubproblem Previously we defined probability as $\prob{A} := \frac{|A|}{|\Omega|}$. Describe a situtation where this fails to produce the correct probability that is not the spinner used in lecture. \spc{4}

\easysubproblem Which definition of probability does the book use and why do you think the authors chose this definition? \spc{4}

\easysubproblem  Give an example of an event whose probability cannot be approximated by the limiting frequency. \spc{1.5}

%\intermediatesubproblem Give an example of a random event involving an object's \qu{propensity} and explain this definition of probability. \spc{2.5}
%
%\easysubproblem Discuss the difference between the \qu{logical} and the \qu{subjective} definition of probability. \spc{4.5}
%
%
%\easysubproblem Explain the difference between \qu{objective} and \qu{epistemic} interpretations of probability. Which definitions fall under these categories? Classify all four of Gillies' definitions in this way. \spc{3.5}

\extracreditsubproblem Who picks $\omega \in \Omega$ \ie the outcome from the set of possible outcomes in the universe? This is an issue we ignored. Discuss your thoughts. \spc{9}

%\extracreditsubproblem Is probability an illusion or is it real? Is randomness a fundamental property of the universe? Discuss your thoughts. \spc{10}

\end{enumerate}

\problem We will be looking into the long term frequency definition here. For this problem, you must have \texttt{R} installed. Please download it from \url{http://cran.r-project.org/} (there are links for Windows, MAC and Linux) and then double-click to open an \texttt{R} console.

\begin{enumerate}

\easysubproblem To calculate combinations, use the \texttt{choose(n,k)} function. Calculate the number of five-card hands from a standard deck by copying the following code into \texttt{R} and then pressing enter:

\begin{knitrout}
\begin{kframe}
\begin{verbatim}
choose(52, 5)
\end{verbatim}
\end{kframe}
\end{knitrout}

Please write down the answer. Is the answer the same as we computed in class? \spc{2.5}

\easysubproblem Verify the probability in class of a ``full house'' by copying the following code into \texttt{R} and then pressing enter:

\begin{knitrout}
\begin{kframe}
\begin{verbatim}
choose(13, 1) * choose(4, 3) * choose(12, 1) * choose(4, 2) / 
  choose(52, 5)
\end{verbatim}
\end{kframe}
\end{knitrout}

Write down the answer as a \textit{percentage}. \spc{3.5}

\intermediatesubproblem We are going to do a little experiment to explore the definition of probability as a limiting frequency. We will be looking at the context of flipping a coin and getting heads. Remember the definition was

\beqn
\prob{\braces{H}} = \limitn \frac{\displaystyle \sum_{i=1}^n \indic{\omega_i \in \braces{H}}}{n}
\eeqn

(where $\indic{T}$ is the \qu{indicator function} which equals 1 when the expression $T$ is true and 0 if the expression $T$ is false). We will run a simulation with large values of $n$. Copy and paste the following code into your \texttt{R} terminal:

\begin{knitrout}
\begin{kframe}
\begin{verbatim}
N = 30000
sims = sample(0:1, N, replace = T)
freqs_by_n = array(NA, N)
for (n in 1 : N){
  freqs_by_n[n] = sum(sims[1:n]) / n
}
plot(10:N, 
  freqs_by_n[10:N], 
  xlim = c(10, N), 
  ylim = c(0.40, 0.60), 
  pch = ".", 
  xlab = "number of samples",
  ylab = "frequency of heads",
  main = "P(H) as a limiting frequency: 30,000 samples")
abline(h = 0.5, col = "blue")
freqs_by_n[N]
#last line placeholder
\end{verbatim}
\end{kframe}
\end{knitrout}

\inred{If the code throws an error, download the PDF to your computer and try copying and pasting again. It does NOT work from a browser window. The PDF must be downloaded.}

The console should have popped up a plot.\footnote{This is a \textit{real} statistical simulation. Each time you run this code it will be different. You can compare plots with your friends but take note that they will not look exactly the same!!} Print this out and attach it to your homework. If you are using \LaTeX, you can include the figure into the PDF.

From the title of the plot and the x and y axes, tell a story about what is going on here \textit{in English}. \spc{3.5}

\easysubproblem What is the limiting frequency of heads after 30,000 coin flips to 3 decimals based on the simulation in the previous problem? (that is the number that appears in the console directly after ``\texttt{$>$ freqs\_by\_n[N]}''). Write it below and comment on its value. \spc{1.5}

\end{enumerate}

\end{document}
